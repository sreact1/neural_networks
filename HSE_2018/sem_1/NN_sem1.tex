%!TEX TS-program = xelatex
\documentclass[12pt, a4paper, oneside]{article}

\usepackage{amsmath,amsfonts,amssymb,amsthm,mathtools}  % пакеты для математики

\usepackage[english, russian]{babel} % выбор языка для документа
\usepackage[utf8]{inputenc} % задание utf8 кодировки исходного tex файла
\usepackage[X2,T2A]{fontenc}        % кодировка

\usepackage{fontspec}         % пакет для подгрузки шрифтов
\setmainfont{Linux Libertine O}   % задаёт основной шрифт документа

\usepackage{unicode-math}     % пакет для установки математического шрифта
\setmathfont[math-style=upright]{Neo Euler} % шрифт для математики

\newcommand{\groupnumber}{1}

\RequirePackage{amsmath}

\newcounter{problem}
\renewcommand{\theproblem}{\arabic{problem}}
\newcommand{\problemname}{Задача}

\newenvironment{problem}{
	\addtocounter{problem}{1}\noindent{\large\bfseries \problemname{} \theproblem}
}{
	\par\bigskip
}

\newenvironment{solution}{
	{\bfseries Решение.}
}{
	\par\bigskip
}

\newenvironment{comment}{
	{\bfseries Комментарии по проверке.}
}{
	\par\bigskip
}

\def\s@tm@cr@s{
	\def\widthin##1{\exmpwidinf=##1\relax}
	\def\widthout##1{\exmpwidouf=##1\relax}
	\def\stretchin##1{\advance\exmpwidinf by ##1\relax}
	\def\stretchout##1{\advance\exmpwidouf by ##1\relax}
	\@ifstar{
		\error Star must not be used in example environment any more
	}
}

\newenvironment{example}[1][]{
	\s@tm@cr@s#1
	\ttfamily\obeylines\obeyspaces\frenchspacing
	\newcommand{\exmp}[2]{
		\begin{minipage}[t]{0.43\textwidth}\rightskip=0pt plus 1fill\relax##1\medskip\end{minipage}&
		\begin{minipage}[t]{0.43\textwidth}\rightskip=0pt plus 1fill\relax##2\medskip\end{minipage}\\
		\hline
	}
	
	\medskip
	\begin{tabular}{|l|l|}
		\hline
		Пример входа & Пример выхода \\
		\hline
	}{
	\end{tabular}
}

\textwidth=440pt
\textheight=620pt
\hoffset=-43pt
\headsep=25pt
\footskip=40pt

\makeatletter

%---------- From package "lastpage" ------------------
\def\lastpage@putlabel{\addtocounter{page}{-1}%
	\immediate\write\@auxout{\string\newlabel{LastPage}{{}{\thepage}}}%
	\addtocounter{page}{1}}
\AtEndDocument{\clearpage\lastpage@putlabel}%
%---------- end of "lastpage" ------------------

\renewcommand{\@oddhead}{%
	\parbox{\textwidth}{%
		\begin{center}
			Введение в нейронные сети \hfill  ВШЭ, осень 2018 г.
			\hrule%
		\end{center}
	}%
}

\makeatother

\usepackage{amsfonts}
\usepackage{expdlist}

\usepackage{xcolor} 
\usepackage{enumitem}
\definecolor{myblue}{rgb}{0, 0.45, 0.70}
\newcommand*{\MyPoint}{\tikz \draw [baseline, fill=myblue,draw=blue] circle (2.5pt);}
\renewcommand{\labelitemi}{\MyPoint}

% расстояние в списках
\setlist[itemize]{parsep=0.4em,itemsep=0em,topsep=0ex}
\setlist[enumerate]{parsep=0.4em,itemsep=0em,topsep=0ex}

\usepackage{tikz, pgfplots}  % язык для рисования графики из latex'a
\usepackage{todonotes}


\begin{document}
	
	
\section*{Тятя! Тятя! Наши сети притащили мертвеца!}


\begin{problem}
Исследователь Арчибальд оценивает модель линейной регрессии $y = \beta \cdot x$. Сегодня ночью он собрал выборку. Из неё он взял два наблюдения: $x_1 = 1, x_2 = 2$, $y_1 = 3, y_2 = 4$. Теперь он хочет оценить $\beta$, сделав два шага стохастического градиентного спуска. Сначала с первым наблюдением, затем со вторым. 

В качестве стартовой точки используется $\beta_0 = 0$. В качестве скорости обучения взяли $\eta = 0.1$. 
\end{problem}


\begin{problem}
Парни очень любят Марго, а Марго любит собирать персептроны и думать по вечерам о их весах и функциях активации. Сегодня она решила разобрать свои залежи из персептронов и как следует упорядочить их. 

\begin{itemize}
	
	\item Для перцептрона 
	
	\begin{center}
		\definecolor{cqcqcq}{rgb}{0.7529411764705882,0.7529411764705882,0.7529411764705882}
		\begin{tikzpicture}[line cap=round,line join=round,x=1.0cm,y=1.0cm]
		\clip(-4,1) rectangle (3.306148366367316,4.5);
		\draw [line width=1.pt] (-3.,4.) circle (0.5cm);
		\draw [line width=1.pt] (-3.152000366859971,1.7166134938755313) circle (0.5cm);
		\draw [line width=1.pt] (-1.,3.)-- (-1.,2.);
		\draw [line width=1.pt] (-1.,2.)-- (1.5,2.);
		\draw [line width=1.pt] (1.5,2.)-- (1.5,3.);
		\draw [line width=1.pt] (1.5,3.)-- (-1.,3.);
		\draw [->,line width=1.pt] (-2.5,4.) -- (-1.0315953503775401,2.702352315488771);
		\draw [->,line width=1.pt] (-2.652000366859971,1.7166134938755313) -- (-1.,2.5);
		\draw [->,line width=1.pt] (1.5,2.5) -- (2.5,2.5);
		\draw (-3.4,2) node[anchor=north west] {$x_1$};
		\draw (-0.8,2.8) node[anchor=north west] {$\max(0,t)$};
		\draw (2.6,2.7) node[anchor=north west] {$y$};
		\draw (-1.8,3.9) node[anchor=north west] {$w_1$};
		\draw (-2.5,2.6) node[anchor=north west] {$w_2$};
		\draw (-3.2,4.25) node[anchor=north west] {$1$};
		\end{tikzpicture}
	\end{center}

нужно подобрать веса так, чтобы он превращал $x_1 = 0$ в $y=1$, а $x_1 = 1$ в $y=0$.
	
	\item  Для перцепторона 
	
	\begin{center}
	\begin{tikzpicture}[line cap=round,line join=round,x=1.0cm,y=1.0cm]
	\clip(-4,0.5) rectangle (3.3,4.5);
	\draw [line width=1.pt] (-3.,4.) circle (0.5cm);
	\draw [line width=1.pt] (-3.,2.5) circle (0.5cm);
	\draw [line width=1.pt] (-3.,1.) circle (0.5cm);
	\draw [line width=1.pt] (-1.,3.)-- (-1.,2.);
	\draw [line width=1.pt] (-1.,2.)-- (1.5,2.);
	\draw [line width=1.pt] (1.5,2.)-- (1.5,3.);
	\draw [line width=1.pt] (1.5,3.)-- (-1.,3.);
	\draw [->,line width=1.pt] (-2.5,4.) -- (-1.0315953503775401,2.702352315488771);
	\draw [->,line width=1.pt] (-2.5,2.5) -- (-1.,2.5);
	\draw [->,line width=1.pt] (-2.5,1.) -- (-1.,2.2833448312560893);
	\draw [->,line width=1.pt] (1.5,2.5) -- (2.5,2.5);
	\draw (-3.3,4.25) node[anchor=north west] {$x_1$};
	\draw (-3.3,2.7) node[anchor=north west] {$x_2$};
	\draw (-3.3,1.26) node[anchor=north west] {$x_3$};
	\draw (-0.67, 2.8) node[anchor=north west] {$\max(0,t)$};
	\draw (2.6747622271028213,2.745742538897389) node[anchor=north west] {$y$};
	\draw (-2.1, 4) node[anchor=north west] {$w_1$};
	\draw (-2.25,3) node[anchor=north west] {$w_2$};
	\draw (-2.4,2.1) node[anchor=north west] {$w_3$};
	\end{tikzpicture}
	\end{center}

Марго хочет по наблюдениям $x$ подобрать такие веса $w_i$, чтобы на выходе получились $y$. 

\begin{center}
	\begin{tabular}{c|c|c|c}
		$x_1$ & $x_2$ & $x_3$ & $y$ \\
		\hline 
		$1$ & $1$ & $2$ & $0.5$\\
		\hline 
		$1$ & $-1$ & $1$ & $0$ \\
	\end{tabular}
\end{center}

	
	\item   Из нескольких перцептронов с неизвестной функцией активации
	\begin{center}
		\begin{tikzpicture}[line cap=round,line join=round,x=1.0cm,y=1.0cm]
		\clip(-4,0.5) rectangle (3,4.5);
		\draw [line width=1.pt] (-3.,4.) circle (0.5cm);
		\draw [line width=1.pt] (-3.,2.5) circle (0.5cm);
		\draw [line width=1.pt] (-3.,1.) circle (0.5cm);
		\draw [line width=1.pt] (-1.,3.)-- (-1.,2.);
		\draw [line width=1.pt] (-1.,2.)-- (1.5,2.);
		\draw [line width=1.pt] (1.5,2.)-- (1.5,3.);
		\draw [line width=1.pt] (1.5,3.)-- (-1.,3.);
		\draw [->,line width=1.pt] (-2.5,4.) -- (-1.0315953503775401,2.702352315488771);
		\draw [->,line width=1.pt] (-2.5,2.5) -- (-1.,2.5);
		\draw [->,line width=1.pt] (-2.5,1.) -- (-1.,2.2833448312560893);
		\draw [->,line width=1.pt] (1.5,2.5) -- (2.5,2.5);
		\draw (-3.3,4.25) node[anchor=north west] {$1$};
		\draw (-3.3,2.7) node[anchor=north west] {$x_1$};
		\draw (-3.3,1.26) node[anchor=north west] {$x_2$};
		\draw (-0.67, 2.8) node[anchor=north west] {$f(t)$};
		\draw (2.6747622271028213,2.745742538897389) node[anchor=north west] {$y$};
		\draw (-2.1, 4) node[anchor=north west] {$w_1$};
		\draw (-2.25,3) node[anchor=north west] {$w_2$};
		\draw (-2.4,2.1) node[anchor=north west] {$w_3$};
		\end{tikzpicture}
	\end{center}

Марго хочет построить нейронную сеть так, чтобы она поделила плоскость на три части следущим образом: 

\begin{center}
\begin{tikzpicture}[line cap=round,line join=round,x=1.0cm,y=1.0cm]
\clip(-3,-5) rectangle (5,-1);
\draw [line width=2.pt] (-3.,4.) circle (0.5cm);
\draw [line width=2.pt] (-3.152000366859971,1.7166134938755313) circle (0.5cm);
\draw [line width=2.pt] (-1.,3.)-- (-1.,2.);
\draw [line width=2.pt] (-1.,2.)-- (1.5,2.);
\draw [line width=2.pt] (1.5,2.)-- (1.5,3.);
\draw [line width=2.pt] (1.5,3.)-- (-1.,3.);
\draw [->,line width=2.pt] (-2.5,4.) -- (-1.0315953503775401,2.702352315488771);
\draw [->,line width=2.pt] (-2.652000366859971,1.7166134938755313) -- (-1.,2.5);
\draw [->,line width=2.pt] (1.5,2.5) -- (2.5,2.5);
\draw (-3.270567410142784,1.9563548477920916) node[anchor=north west] {$x_1$};
\draw (-0.17178835933248715,2.8036772444980356) node[anchor=north west] {$\max(0,t)$};
\draw (2.6727939724660277,2.7552588218291243) node[anchor=north west] {$y$};
\draw (-1.7695963074065464,3.8809871488813075) node[anchor=north west] {$w_1$};
\draw (-2.217466717093972,2.6342127651568465) node[anchor=north west] {$w_2$};
\draw (-3.088998325134368,4.2562299245653685) node[anchor=north west] {$1$};
\draw [->,line width=2.pt] (1.,-5.) -- (1.,-1.);
\draw [->,line width=2.pt] (-2.,-3.) -- (4.,-3.);
\draw [line width=2.pt,dash pattern=on 3pt off 3pt] (0.,-1.)-- (0.,-5.);
\draw [line width=2.pt,dash pattern=on 3pt off 3pt] (2.,-1.)-- (2.,-5.);
\draw (-1,-2) node[anchor=north west] {$1$};
\draw (2.5,-2) node[anchor=north west] {$1$};
\draw (1.25,-2) node[anchor=north west] {$0$};
\end{tikzpicture}
\end{center} 
	
\item  На плоскости проведены две прямые $x_1 + x_2 = 1$ и $x_1 - x_2 = 1$. Соберите из перцептронов из предыдущего пункта нейросетку, которая поделит плоскость следущим образом: 

\begin{center}
\begin{tikzpicture}[line cap=round,line join=round,x=1.0cm,y=1.0cm]
\clip(-2,-5) rectangle (4,-0.5);
\draw [line width=2.pt] (-3.,4.) circle (0.5cm);
\draw [line width=2.pt] (-3.152000366859971,1.7166134938755313) circle (0.5cm);
\draw [line width=2.pt] (-1.,3.)-- (-1.,2.);
\draw [line width=2.pt] (-1.,2.)-- (1.5,2.);
\draw [line width=2.pt] (1.5,2.)-- (1.5,3.);
\draw [line width=2.pt] (1.5,3.)-- (-1.,3.);
\draw [->,line width=2.pt] (-2.5,4.) -- (-1.0315953503775401,2.702352315488771);
\draw [->,line width=2.pt] (-2.652000366859971,1.7166134938755313) -- (-1.,2.5);
\draw [->,line width=2.pt] (1.5,2.5) -- (2.5,2.5);
\draw (-3.270567410142784,1.9563548477920916) node[anchor=north west] {$x_1$};
\draw (-0.17178835933248715,2.8036772444980356) node[anchor=north west] {$\max(0,t)$};
\draw (2.6727939724660277,2.7552588218291243) node[anchor=north west] {$y$};
\draw (-1.7695963074065464,3.8809871488813075) node[anchor=north west] {$w_1$};
\draw (-2.217466717093972,2.6342127651568465) node[anchor=north west] {$w_2$};
\draw (-3.088998325134368,4.2562299245653685) node[anchor=north west] {$1$};
\draw [->,line width=2.pt] (1.,-5.) -- (1.,-1.);
\draw [->,line width=2.pt] (-2.,-3.) -- (4.,-3.);
\draw (2.0191452664357303,-1.251365654023268) node[anchor=north west] {$1$};
\draw (-0.3170436273392198,-2.4134077980771345) node[anchor=north west] {$0$};
\draw [line width=2.pt,dash pattern=on 3pt off 3pt] (0.,-5.)-- (4.,-1.);
\draw [line width=2.pt,dash pattern=on 3pt off 3pt] (0.,-1.)-- (4.,-5.);
\draw (1.910203815430681,-4.0112157461512) node[anchor=north west] {$0$};
\draw (3.2417104388257303,-2.4013031924099066) node[anchor=north west] {$0$};
\end{tikzpicture}
\end{center}
\end{itemize}
\end{problem}


\begin{problem}

Маша услышала про машин лёрнинг и решила, что она и есть та самая Маша, которой этот лёрнинг принадлежит. Теперь она собирается обучить нейронную сеть для решения задачи регрессии, На вход в ней идёт $12$ переменных, в сетке есть $3$ скрытых слоя. В пером слое $300$ нейронов, во втором $200$, в третьем $100$. 

\begin{itemize}
	\item[a)] Сколько параметров предстоит оценить Маше?  Сколько наблюдений вы бы на её месте использовали? 
	\item[b)] Что Маша должна сделать с внешним слоем, если она собирается решать задачу классификации на два класса и получать на выходе вероятность принадлежности к первому классу? 
	\item[c)]  Что делать Маше, если она хочет решать задачу классификации на $K$ классов? 
\end{itemize}
\end{problem}


\begin{problem}

Как-то раз Вовочка решал задачу классификации. С тех пор у него в кармане завалялась нейросеть: 

\begin{center}
\begin{tikzpicture}[scale = 1.5, line cap=round,line join=round,x=1.0cm,y=1.0cm]
\clip(-4.624679143882289,1.3686903642723092) rectangle (3.8521836267384844,4.8154607149701345);
\draw [line width=2.pt] (-2.5,2.5) circle (0.5cm);
\draw [line width=2.pt] (-0.5,2.5) circle (0.5cm);
\draw [line width=2.pt] (1.5,2.5) circle (0.5cm);
\draw [->,line width=2.pt] (-3.5,4.) -- (-2.831496141146421,2.874313115459547);
\draw [->,line width=2.pt] (-4.,2.5) -- (-3.,2.5);
\draw [->,line width=2.pt] (-2.,2.5) -- (-1.,2.5);
\draw [->,line width=2.pt] (0.,2.5) -- (1.,2.5);
\draw [->,line width=2.pt] (2.,2.5) -- (3.,2.5);
\draw [->,line width=2.pt] (-1.5,4.) -- (-0.8294354120380936,2.8761280490674572);
\draw [->,line width=2.pt] (0.5,4.) -- (1.1775032003746246,2.8820939861230355);
\draw (-3.686865302879703,4.4622580995276016) node[anchor=north west] {$1$};
\draw (-1.67726421501702,4.474437500060103) node[anchor=north west] {$1$};
\draw (0.2957986712481599,4.267387691007583) node[anchor=north west] {1};
\draw (-4.320194130569761,2.805859627107445) node[anchor=north west] {$x$};
\draw (3.1214195947884176,2.7693214255099416) node[anchor=north west] {$y$};
\draw (-3.7112241039447054,3.07380643882247) node[anchor=north west] {$w_1^1$};
\draw (-3.114433477852151,3.9750820782275555) node[anchor=north west] {$w_0^1$};
\draw (-1.7138024166145234,3.122524040952475) node[anchor=north west] {$w_1^2$};
\draw (-1.1535499921194723,4.13341428515007) node[anchor=north west] {$w_0^2$};
\draw (1.0387421037307276,4.109055484085068) node[anchor=north west] {$w_0^3$};
\draw (0.2836192707156588,3.0859858393549713) node[anchor=north west] {$w_1^3$};
\end{tikzpicture}
\end{center} 

В качестве функции активации используется сигмоид: $f(t) = \frac{e^t}{1 + e^t}$.  Есть два наблюдения: $x_1 = 1, x_2 = 5$, $y_1 =1$, $y_2 = 0$. Скорость обучения $\gamma = 1$. В качестве инициализации взяты нулевые веса. Как это обычно бывает, Вовочка обнаружил её в своих штанах после стирки и очень обрадовался.  Теперь он собирается сделать два шага стохастического градиентного спуска, используя алгоритм обратного распространения ошибки. Помогите ему. 
\end{problem}

\end{document}