%!TEX TS-program = xelatex
\documentclass[notes,12pt, aspectratio=169]{beamer}

\usepackage{amsmath,amsfonts,amssymb,amsthm,mathtools}  % пакеты для математики
\usepackage{minted}

\usepackage[english, russian]{babel} % выбор языка для документа
\usepackage[utf8]{inputenc} % задание utf8 кодировки исходного tex файла
\usepackage[X2,T2A]{fontenc}        % кодировка

\usepackage{fontspec}         % пакет для подгрузки шрифтов
\setmainfont{Helvetica}  % задаёт основной шрифт документа

% why do we need \newfontfamily:
% http://tex.stackexchange.com/questions/91507/
\newfontfamily{\cyrillicfonttt}{Helvetica}
\newfontfamily{\cyrillicfont}{Helvetica}
\newfontfamily{\cyrillicfontsf}{Helvetica}

\usepackage{unicode-math}     % пакет для установки математического шрифта
% \setmathfont{Neo Euler} % шрифт для математики

\usepackage{polyglossia}      % Пакет, который позволяет подгружать русские буквы
\setdefaultlanguage{russian}  % Основной язык документа
\setotherlanguage{english}    % Второстепенный язык документа

% Шрифт для кода
\setmonofont[Scale=0.85]{Monaco}
\usepackage{verbments}

\usepackage{pgfpages}
% These slides also contain speaker notes. You can print just the slides,
% just the notes, or both, depending on the setting below. Comment out the want
% you want.
%\setbeameroption{hide notes} % Only slide
%\setbeameroption{show only notes} % Only notes
%\setbeameroption{show notes on second screen=right} % Both

\usepackage{array}

\usepackage{tikz}
\usepackage{verbatim}
\setbeamertemplate{note page}{\pagecolor{yellow!5}\insertnote}
\usetikzlibrary{positioning}
\usetikzlibrary{snakes}
\usetikzlibrary{calc}
\usetikzlibrary{arrows}
\usetikzlibrary{decorations.markings}
\usetikzlibrary{shapes.misc}
\usetikzlibrary{matrix,shapes,arrows,fit,tikzmark}

\usepackage{hyperref}
\usepackage{lipsum}
\usepackage{multimedia}
\usepackage{multirow}
\usepackage{dcolumn}
\usepackage{bbm}
\newcolumntype{d}[0]{D{.}{.}{5}}

\usepackage{changepage}
\usepackage{appendixnumberbeamer}
\newcommand{\beginbackup}{
   \newcounter{framenumbervorappendix}
   \setcounter{framenumbervorappendix}{\value{framenumber}}
   \setbeamertemplate{footline}
   {
     \leavevmode%
     \hline
     box{%
       \begin{beamercolorbox}[wd=\paperwidth,ht=2.25ex,dp=1ex,right]{footlinecolor}%
%         \insertframenumber  \hspace*{2ex} 
       \end{beamercolorbox}}%
     \vskip0pt%
   }
 }
\newcommand{\backupend}{
   \addtocounter{framenumbervorappendix}{-\value{framenumber}}
   \addtocounter{framenumber}{\value{framenumbervorappendix}} 
}

% для имитации питоновского синтаксиса 
\newcommand{\pgr}[1]{{\color{green} \textbf{#1}}}


%%%%%%%%%% Работа с картинками %%%%%%%%%
\usepackage{graphicx}                  % Для вставки рисунков
\usepackage{graphics}
\graphicspath{{images_2/}}    % можно указать папки с картинками
\usepackage{wrapfig}                   % Обтекание рисунков и таблиц текстом

\usepackage[space]{grffile}
\usepackage{booktabs}

% These are my colors -- there are many like them, but these ones are mine.
\definecolor{blue}{RGB}{0,114,178}
\definecolor{red}{RGB}{213,94,0}
\definecolor{yellow}{RGB}{240,228,66}
\definecolor{green}{RGB}{0,128, 0}

\hypersetup{
  colorlinks=false,
  linkbordercolor = {white},
  linkcolor = {blue}
}


%% I use a beige off white for my background
\definecolor{MyBackground}{RGB}{255,253,218}

%% Uncomment this if you want to change the background color to something else
%\setbeamercolor{background canvas}{bg=MyBackground}

%% Change the bg color to adjust your transition slide background color!
\newenvironment{transitionframe}{
  \setbeamercolor{background canvas}{bg=yellow}
  \begin{frame}}{
    \end{frame}
}

\setbeamercolor{frametitle}{fg=blue}
\setbeamercolor{title}{fg=black}
\setbeamertemplate{footline}[frame number]
\setbeamertemplate{navigation symbols}{} 
\setbeamertemplate{itemize items}{-}
\setbeamercolor{itemize item}{fg=blue}
\setbeamercolor{itemize subitem}{fg=blue}
\setbeamercolor{enumerate item}{fg=blue}
\setbeamercolor{enumerate subitem}{fg=blue}
\setbeamercolor{button}{bg=MyBackground,fg=blue,}


% If you like road maps, rather than having clutter at the top, have a roadmap show up at the end of each section 
% (and after your introduction)
% Uncomment this is if you want the roadmap!
% \AtBeginSection[]
% {
%    \begin{frame}
%        \frametitle{Roadmap of Talk}
%        \tableofcontents[currentsection]
%    \end{frame}
% }
\setbeamercolor{section in toc}{fg=blue}
\setbeamercolor{subsection in toc}{fg=red}
\setbeamersize{text margin left=1em,text margin right=1em} 

% списки, которые растягиваются на всю величину слайда 
\newenvironment{wideitemize}{\itemize\addtolength{\itemsep}{10pt}}{\enditemize}



\title[]{\textcolor{blue}{Практический анализ данных и машинное обучение: искусственные нейронные сети}}
\author{Ульянкин Филипп}
\date{\today}


\begin{document}

%%% TIKZ STUFF
\tikzset{   
        every picture/.style={remember picture,baseline},
        every node/.style={anchor=base,align=center,outer sep=1.5pt},
        every path/.style={thick},
        }
\newcommand\marktopleft[1]{%
    \tikz[overlay,remember picture] 
        \node (marker-#1-a) at (-.3em,.3em) {};%
}
\newcommand\markbottomright[2]{%
    \tikz[overlay,remember picture] 
        \node (marker-#1-b) at (0em,0em) {};%
}
\tikzstyle{every picture}+=[remember picture] 
\tikzstyle{mybox} =[draw=black, very thick, rectangle, inner sep=10pt, inner ysep=20pt]
\tikzstyle{fancytitle} =[draw=black,fill=red, text=white]
%%%% END TIKZ STUFF

% Title Slide
\begin{frame}
\maketitle
\centering Свёрточные нейросетки
\end{frame}



\begin{frame}{Agenda}
\begin{wideitemize}

\item Что такое свёртка или как видит компьютер 

\item Снижаем размерность картинки 

\item Собираем свою собственную свёрточную нейросеть

\end{wideitemize} 
\end{frame}


\begin{frame}{Картинка — тензор}
\begin{itemize}
	\item Каждая картинка — это матрица из пикселей
	
	\item Каждый пиксель обладает яркостью по шкале от $0$ до $255$ 
	
	\item Цветное изображение имеет три канала пикселей: красный, зелёный и синий
	
\end{itemize}

\begin{center}
	\includegraphics[width=.7\linewidth]{pixels.png}
\end{center}
\end{frame}

% сделать на следущей итерации курса более подробные слайды про тензоры 


\begin{frame}{Обычная сетка}
\begin{center}
	\includegraphics[width=.6\linewidth]{not_conv.png}
\end{center}

\begin{wideitemize}
	\item Очень много весов 
	\item Теряется информация о взаимном расположении пикселей
	\item Изображение в разных местах картинки даёт разные веса
\end{wideitemize}
\end{frame}


\begin{frame}
\begin{center}
	\includegraphics[width=.8\linewidth]{cat_1.png}
\end{center}
\end{frame}


\begin{frame}
\begin{center}
	\includegraphics[width=.8\linewidth]{cat_2.png}
\end{center}
\end{frame}


\begin{frame}
\begin{center}
	\includegraphics[width=.8\linewidth]{cat_3.png}
\end{center}
\end{frame}


\begin{frame}{Обычная сетка}
\begin{wideitemize}
	\item Хотим, чтобы информация не терялась
	\item Хотим одинаковые веса $\Rightarrow$ {\color{red} свёртка} 
\end{wideitemize}
\end{frame}


\begin{transitionframe}
	\begin{center}
		\Huge Свёртка в питоновской тетрадке
	\end{center}
\end{transitionframe}


\begin{frame}{Свёртка}
\begin{center}
	\includegraphics[width=.8\linewidth]{conv_1.png}
\end{center}
\end{frame}


\begin{frame}{Свёртка}
\begin{center}
	\includegraphics[width=.8\linewidth]{conv_2.png}
\end{center}
\end{frame}


\begin{frame}{Свёртка инвариантна к расположению}
\begin{center}
	\includegraphics[width=.8\linewidth]{conv_3.png}
\end{center}
\end{frame}


\begin{frame}{Свёртка инвариантна к расположению}
\begin{center}
	\includegraphics[width=.8\linewidth]{conv_4.png}
\end{center}
\end{frame}



% вопросик про линии 


\begin{frame}{Свёрточный слой}
\begin{center}
	\includegraphics[width=.7\linewidth]{conv_layer.png}
\end{center}
\end{frame}


\begin{frame}{Свёрточный слой}
\begin{center}
	\includegraphics[width=.7\linewidth]{conv_layer_2.png}
\end{center}
\end{frame}

% при переработке презы впихнуть сюда слайды про backpropagation и перевести тикущие на русский язык


\begin{frame}{Свёрточный слой}
\begin{wideitemize}
	\item  Слой действует одинаково для каждого участка картинки, в отличие от полносвязного 
	\item  Нужно оценивать меньшее количество параметров
	\item  Слой учитывает взаимное расположение пикселей
\end{wideitemize}
\end{frame}


\begin{frame}{Слой из нескольких фильтров}
\begin{center}
	\includegraphics[width=.5\linewidth]{convolution3.png}
\end{center}
\end{frame}


\begin{transitionframe}
	\begin{center}
		\Huge  Снижение размерности
	\end{center}
\end{transitionframe}


\begin{frame}{Дополнение (Padding)}
\begin{center}
	\includegraphics[width=.5\linewidth]{padding.png}
\end{center}
\begin{wideitemize}
	\item Padding используют, чтобы пространственная размерность картинки не уменьшалась после применения свертки или уменьшалась не так быстро
\end{wideitemize}
\end{frame}


\begin{frame}{Сдвиг (Stride)}
\begin{center}
	\includegraphics[width=.7\linewidth]{stride.png}
\end{center}
\begin{wideitemize}
	\item Пиксели локально скоррелированы — соседние пиксели, как правило, не сильно отличаются друг от друга	
	\item  Если будем делать свёртка идёт с каким-то шагом, сэкономим мощности компьютера (число весов) и не потеряем в информации
	\item  Очень агрессивная стратегия снижения размерности картинки 
\end{wideitemize}
\end{frame}


\begin{frame}{Пулинг (Pooling)}
\begin{center}
	\includegraphics[width=.7\linewidth]{pooling.png}
\end{center}
\begin{wideitemize}
	\item  Будем считать внутри какого-то окна максимум или среднее и сворачивать размерность, пользуясь локальной скоррелированностью 
\end{wideitemize}
\end{frame}


\begin{frame}{Простейшая CNN}
\begin{center}
	\includegraphics[width=.7\linewidth]{lenet.png}
\end{center}
\begin{wideitemize}
	\item  Нейросетка LeNet-5  (1998) для распознавания рукописных цифр из датасета MNIST
\end{wideitemize}
\end{frame}


\begin{frame}{Data augmentation}
\begin{wideitemize}
	\item  В сетке может быть миллионы параметров!
	\item  Естественная регуляризация, дополнительная регуляризация, генерация новых данных (data augmentation) 
	\item Генерируем новые цвета, сдвигаем, искажаем и тп 
\end{wideitemize}

\begin{center}
	\includegraphics[width=.7\linewidth]{cat_data_augmentation.png}
\end{center}

\vfill 
\footnotesize{\url{https://blog.keras.io/building-powerful-image-classification-models-using-very-little-data.html}}
\end{frame}


\begin{frame}{Что выучивают нейросети}
\begin{center}
	\includegraphics[width=.9\linewidth]{cnn_vis.png}
\end{center}
\vfill 
\small{\url{https://arxiv.org/pdf/1311.2901.pdf}}
\end{frame}


\begin{transitionframe}
	\begin{center}
		\Huge  Собираем свою собственную CNN
	\end{center}
\end{transitionframe}


%\begin{transitionframe}
%	\begin{center}
%		\Huge  Как нейросети рвали ImageNet
%	\end{center}
%\end{transitionframe}
%
%
%\begin{frame}{ImageNet}
%\begin{itemize} 
%	\item  $1000$ классов, около $1.2$ изображений с проставленными метками, человек распознаёт изображения с точностью $95\%$
%\end{itemize}
%
%\begin{center}
%	\includegraphics[width=.7\linewidth]{imagenet.png}
%\end{center}
%\end{frame}
%
%
%\begin{frame}{ImageNet}
%\begin{center}
%	\includegraphics[width=.8\linewidth]{imagenet_error_2014.png}
%\end{center}
%\end{frame}


\end{document}
